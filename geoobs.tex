\documentclass{patmorin}
\listfiles
\usepackage[utf8]{inputenc}
\usepackage{microtype}
\usepackage{amsthm,amsmath,graphicx}
\usepackage{pat}
\usepackage[letterpaper]{hyperref}
\usepackage[table,dvipsnames]{xcolor}
\definecolor{linkblue}{named}{Blue}
\hypersetup{colorlinks=true, linkcolor=linkblue,  anchorcolor=linkblue,
citecolor=linkblue, filecolor=linkblue, menucolor=linkblue,
urlcolor=linkblue} 
\setlength{\parskip}{1ex}

\title{\MakeUppercase{Geodesic Obstacle Representations}}

\author{Pat Morin and Friends}%

%\usepackage[mathlines]{lineno}
%\linenumbers
%\setlength{\linenumbersep}{2.5cm}
%\rightlinenumbers
%\linenumbers
%\newcommand*\patchAmsMathEnvironmentForLineno[1]{%
%  \expandafter\let\csname old#1\expandafter\endcsname\csname #1\endcsname
%  \expandafter\let\csname oldend#1\expandafter\endcsname\csname end#1\endcsname
%  \renewenvironment{#1}%
%     {\linenomath\csname old#1\endcsname}%
%     {\csname oldend#1\endcsname\endlinenomath}}% 
%\newcommand*\patchBothAmsMathEnvironmentsForLineno[1]{%
%  \patchAmsMathEnvironmentForLineno{#1}%
%  \patchAmsMathEnvironmentForLineno{#1*}}%
%\AtBeginDocument{%
%\patchBothAmsMathEnvironmentsForLineno{equation}%
%\patchBothAmsMathEnvironmentsForLineno{align}%
%\patchBothAmsMathEnvironmentsForLineno{flalign}%
%\patchBothAmsMathEnvironmentsForLineno{alignat}%
%\patchBothAmsMathEnvironmentsForLineno{gather}%
%\patchBothAmsMathEnvironmentsForLineno{multline}%
%}


\DeclareMathOperator{\ob}{obs}
\DeclareMathOperator{\planeobs}{plane-obs}

\pagenumbering{roman}
\begin{document}
\begin{titlepage}
\maketitle

\begin{abstract}
  TBD
\end{abstract}
\end{titlepage}

\tableofcontents

\newpage


\section{Introduction}
\pagenumbering{arabic}

An obstacle representation of an (undirected simple) graph $G$ is
pair $(\varphi, S)$ where $\varphi:V(G)\to\R^2$ maps vertices of $G$
to distinct points in $\R^2$ and $S$ is a set of connected subsets of
$\R^2$ with the property that, for every $u,w\in V(G)$, $uw\in E(G)$
if and only if the line segment $\varphi(u)\varphi(w)$ is disjoint from
$\cup S$.  The elements of $S$ are called \emph{obstacles}.

It is easy to see that every graph $G$ has an obstacle representation:
Take any $\varphi$ that does not map three vertices of $G$ onto a single
line. This gives a straight-line drawing of $G$ whose edges form an
arrangement of line segments.  Now take $S$ to consist of the open faces
in this arrangement.  Since every graph has an obstacle representation,
this defines a natural graph parameter called the \emph{obstacle number}:
\[
     \ob(G) = \min\{|S| :\text{$(\varphi, S)$ is an obstacle representation of $G$}\} \enspace .
\]
Since their introduction by So and So, obstacle numbers have been studied
extensively, with the main goal of bounding the obstacle numbers of
various classes of graphs.  It is known that every $n$-vertex graph has
obstacle number $O(n\log n)$ \cite{X} and some $n$-vertex graphs have
obstacle number $\Omega(n/(\log\log n)^2)$ \cite{Y}.  For planar graphs,
our understanding is muddier: There exists planar graphs with obstacle
number 2 \emph{X} (the icosahedron is an example), but the best upper
bound on the obstacle number of an $n$-vertex planar graph is $O(n)$.

For planar graphs there is also a natural notion of a \emph{plane obstacle
representation} $(\varphi, S)$ which is an obstacle representation
in which $\varphi$ defines a plane straight-line embedding of $G$.
This leads to a \emph{plane obstacle number}
\[
    \planeobs(G) = \min\{|S| :\text{$(\varphi, S)$ is a plane obstacle representation of $G$}\} \enspace .
\]
Using Euler's Formula, it is not hard to see that the plane obstacle
number of any $n$-vertex planar graph is $O(n)$: Let $\varphi$ define
any plane drawing of $G$ with no three vertices collinear and take $S$
to be the set of open faces in this drawing.  Since an $n$-vertex planar
graph has at most $2n-4$ faces, this implies $\planeobs(G)\le 2n-4$.
A lower bound of $\Omega(n)$ is also not difficult:  Any plane drawing
of the $\sqrt{n}\times\sqrt{n}$ grid $G_{\sqrt{n}\times\sqrt{n}}$
has at least $n-2\sqrt{n}$ bounded faces. Each of these faces has
at least four vertices and therefore requires at least one obstacle,
so $\planeobs(G_{\sqrt{n}\times\sqrt{n}})\ge n-2\sqrt{n}$.  So and so
\cite{X} have nailed the leading constant by showing that every planar
graph has obstacle number at most $n-x$ and some planar graphs have
obstacle number $n-x$.

\subsection{A Generalization}

A \emph{curve} over a set $X$ is a function $f:[0,1]\to X$. We denote the
set of curves over $X$ as $\mathcal{C}(X)$.  We will sometimes identify a
curve with its image by making statements like $f$ is contains in $X'$,
which means $f(x)\in X'$ for all $x\in[0,1]$.  A \emph{path space} is
a triple $(X,d,\rho)$, where $(X,d)$ is a metric space and $\rho:X^2\to
\mathcal{C}(X)$ is such that, for every $p,q\in X^2$, every curve $f\in
\rho(p,q)$, and every $0<t<1$,
\begin{enumerate}
  \item $f(0)=p$ and $f(1)=q$;
  \item $d(f(0),f(t)) + d(f(t),f(1)) = d(f(0),f(1))$; 
  \item the curve $g(x)=f(1-x)\in \rho(q,p)$;
  \item the curve $g(x)=f(x)/t$, $x\in[0,t]$ is in $\rho(f(0),f(t))$; and
  \item the curve $h(x)=f(x-t)/(1-t)$, $x\in[t,1]$ is in $\rho(f(t),f(0))$ \enspace .
\end{enumerate} 

For example, we can take $X=\R^2$, $d$ to be the Euclidean distance,
and, for every $p,q\in\R^2$, $\rho(p,q)$ is the set containing only the
line segment joining $s$ and $t$.  A more interesting example (and one
we will study) is when we take $X=\R^2$, $d$ to be the $L_1$ distance
and $\rho(p,q)$ to be the set of all x-y-monotone paths from $p$ to $q$.

A subset $X'\subset X$ in a path space $(X,d,\rho)$ is \emph{connected}
if, for every $p,q\in X'$, there is a sequence $p=x_0,\ldots,x_r=q$ such
that, for all $i\in\{0,\ldots,r-1\}$, $\rho(x_i,x_{i+1})$ contains a path
that is contained in $X'$.\footnote{This definition of connectivity is
not sufficiently detailed to cover some pathological examples, but is
sufficient for everything in this paper.}

With these definitions in hand, we are ready to define a generalization of
obstacle representations.  For a graph $G$, an \emph{$(X,d,\rho)$ obstacle
representation} of $G$ is a pair $(\varphi, S)$ where $\varphi:V(G)\to
X$ is a one-to-one mapping and $S$ is a set of connected subsets of $X$
with the property that, for every $u,w\in V(G)$, $uw\in E(G)$ if and
only if $\rho(\varphi(u),\varphi(w)$ contains a curve that is disjoint
from $\cup S$. That is, there exists $f\in\rho(\varphi(u),\varphi(w)$
such that $f(x)\not\in S$ for all $x\in[0,1]$.

Note that, for a given $(X,d,\rho)$ it is not at all obvious whether every
graph $G$ has an \emph{$(X,d,\rho)$ obstacle representation}.  There are
obvious examples where this is not the case, such as (for example)
when $X$ is finite.  Later, we will show some less obvious examples.

The notion of plane obstacle representation also has a natural
generalization.  An \emph{$(X,d,\rho)$ obstacle representation} of $G$
is \emph{non-crossing} if there is a function $c:E(G)\to \mathcal{C}(X)$
such that, for every $uw,xz\in E(G)$ with $\{u,w\}\cap \{x,z\}=\emptyset$,
(i)~$c(uw)\in \rho(uw)$; and (ii)~$c(uw)$ is disjoint from $c(xz)$.

In this paper, we mostly focus on non-crossing $(X,d,\rho)$
obstacle representations where $X=\R^2$ and $d$ is some
polygonal distance function.  Let $d_k$ denote the distance
function over $\R^2$ where a ``circle'' is a regular $k$-gon.
(Formally, $d_k(p,q)=\min\left\{\sum_{i=1}^{k-1}|a_i|:
q-p=\sum_{i=0}^{k-i}a_iv_i\right\}$, where $v_i=(\cos(i2\pi/k),
\sin(i2\pi/k))$.)  and let $\rho_k(p,q)$ be the set of all shortest
paths from $p$ to $q$ under the distance function $d_k$.  We prove the
following results:
\begin{enumerate}
   \item For any integer $r$, there is a graph that does not have a
     $(\R^2,d_r,\rho_r)$ obstacle representation.
   \item Every planar graph of treewidth at most 2 (and hence every
     outerplanar graph) has a
     non-crossing $(\R^2,d_4,\rho_4)$ obstacle representation.
   \item Not every planar 3-tree has a 
     non-crossing $(\R^2,d_4,\rho_4)$ obstacle representation.
   \item Not every planar 4-connected triangulation has a 
     non-crossing $(\R^2,d_4,\rho_4)$ obstacle representation.
   \item Every planar graph of treewidth at most 3 has a 
     non-crossing $(\R^2,d_6,\rho_6)$ obstacle representation.
   \item Every planar graph has a non-crossing
     $(\R^2,d_{10},\rho_{10})$ obstacle representation.
\end{enumerate}

\subsection{Related Work}

Biedl and Mehrabi study so-called grid obstacle representations \cite{X}...

\section{Preliminaries}

Here we present some definitions and observations that are useful
throughout the rest of the paper.  

\subsection{$d_k$ obstacle embeddings}

We say that a set of points in $\R^2$ is $d_k$-generic, if it
contains no pair of distinct $p$ and $q$ such that $p-q=
t(\cos(i2\pi/k),\sin(i2\pi/k)$ for some integer $i$.

\begin{obs}\obslabel{generic}
  If $G$ has a $(\R^2,d_k,\rho_k)$ obstacle representation, then it
  has an $(\R^2,d_k,\rho_k)$ obstacle representation $(\varphi, S)$
  in which $\{\varphi(u):u\in V(G)\}$ is $d_k$-generic.
\end{obs}

\begin{proof}
   Given an obstacle representation that is non $d_k$-generic, we can
   perform a sufficiently small rotation of the entire representation
   so that it becomes $d_k$-generic.
\end{proof}

\begin{obs}
  If $G$ has a $(\R^2,d_k,\rho_k)$ obstacle representation, then it has
  an $(\R^2,d_k,\rho_k)$ 
\end{obs}

For any $k\in\N$, $i\in\{0,\ldots,k-1\}$, and $u\in\R^2$, define
the $i$th $k$-sector of $u$ as
\[
Q^{k}_i(u) = \{ u+r: r\in\R^2,\,\lfloor i2\pi/k < \angle(u+(1,0),u,r) < (i+1)2\pi/k \}
\]
Where $\angle (a,b,c)\in[0,2\pi)$ denotes the counterclockwise angle
between the segments $ab$ and $cb$.  A curve $f:[0,1]\to\R^2$ is
$k$-monotone in direction $i$ if $f(b)\in Q^k_i(f(a))$ for all $0 \le
a < b \le 1$.  We say that $f$ is $k$-monotone if there exists some
$i\in\{0,\ldots,d-1\}$ such that $f$ is $k$-monotone in direction $i$.

\begin{obs}
   Every $k$-monotone curve with endpoints $p$ and $q$ is in $\rho_k(p,q)$.
\end{obs}

\begin{obs}\obslabel{monotone}
   A graph $G$ has a non-crossing $(\R^2,d_k,\rho_k)$ obstacle
   representation if and only $G$ has a plane embedding such that
\begin{enumerate}
   \item every edge of $G$ is embedded as a $k$-monotone curve; and
   \item for every $u,w\in V(G)$, the embedding contains a $k$-monotone path between $u$ and $w$ if and only if $uw\in E(G)$.
\end{enumerate}
\end{obs}

\begin{proof}
   First we show that the embedding of $G$ gives an $(\R^2,d_k,\rho_k)$
   obstacle representation $(\varphi, S)$. For this, we take $\varphi(u)$
   to be the point that represents $u$ in the embedding of $G$ and we
   take $S$ to be the set of (open) faces in the embedding.

   If $uw\in E(G)$ then the embedding $f$ of $uw$ is a $k$-monotone
   curve that avoids $\cup S$.  Thus, $f\in\rho(u,w)$ and $f\subseteq
   \R^2\setminus{\cup S}$.  On the other hand, if $uw\not\in G$,
   then any path $f$ from $u$ to $w$ in $\R^2\setminus{\cup S}$ is
   a path in the embedding of $G$ and is therefore not $k$-monotone,
   so $f\not\in \rho(u,w)$.

   Next we argue that a $(\R^2,d_k,\rho_k)$ non-crossing obstacle
   representation $(\varphi, S)$ gives the necessary plane embedding
   of $G$.  The embedding we choose maps each vertex $u\in V(G)$ onto
   $\varphi(u)$.  By \obsref{generic} we can assume that the resulting
   point set is $d_k$-generic.  


   Recall that a non-crossing obstacle representation must have a
   function $c:E(G)\to\mathcal{C}(\R^2)$ that maps edges of $G$ onto
   curves joining their endpoints.  In our case, the edge between two
   vertices mapped onto points $p$ and $q$ is mapped onto a curve in
   $\rho_k(p,q)$ and, since $\{p,q\}$ is $d_k$-generic, this curve is
   (or can be made) $k$-monotone.

   The resulting embedding has all the properties required except that
   it may not be plane.  In particular, we may have edges $ux$ and $uz$ 
   such that $c(ux)$ and $c(uz)$ cross each other.
   In this case, we observe that both
   curves are $k$-monotone in direction $i$ for the same value of $i$.
   This makes it easy to eliminate crossings by repeatedly swapping
   segments of the curves and making local modifications around the
   crossings.  Repeating this gives the desired embedding of $G$.
\end{proof}

\obsref{monotone} allows us to focus our attention on embeddings
of graphs that have Properties~1 and 2.  We call these \emph{$d_k$-obstacle
embeddings}.

\subsection{Treewidth}

A \emph{$k$-tree} is any graph that can be obtained in the following
manner:  We begin with a clique on $k+1$ vertices and then we repeatedly
select a subset of the vertices that form a $k$-clique $K$ and add a
new vertex adjacent to every element in $K$.  The class of $k$-trees is
exactly the set of edge-maximal graphs of treewidth $k$.  We will make
use of the following lemma, due to Dujmovi\'c and Wood \cite[Lemma~Y]{X}
in some recursive embeddings of partial 2- and 3-trees.

\begin{lem}\lemlabel{dujwood}
   Every $k$-tree is either a clique on $k+1$ vertices or it contains a non-empty independent set $S$ and a vertex $u\not\in S$, such that:
\begin{enumerate}
   \item $G\setminus S$ is a $k$-tree;
   \item $\deg_{G\setminus S}(u)=k$; and 
   \item every element in $S$ is adjacent to $u$ and $k-1$ elements of
   $N_{G\setminus S}(u)$.
\end{enumerate}
\end{lem}

A graph $G$ is called a \emph{partial $k$-tree} if it is a subgraph of
some $k$-tree.  The class of partial $k$-trees is exactly the class of
graphs of treewidth at most $k$.  The class of partial 1-trees is the
class of forests.  All partial 2-trees are planar, every outerplanar graph
is a partial 2-tree, and every series-parallel graph is a partial 2-tree.
Not every 3-tree, and hence not every partial 3-trees is planar, since
$K_{3,3}$ is a 3-tree.

\section{Grid Obstacle Representations}

In this section we focus on $d_4$ obstacle embeddings.

\begin{thm}
  Every partial 2-tree has a straight-line $d_4$ obstacle embedding.
\end{thm}

\begin{proof}
  Let $G$ be a partial 2-tree. We can, without loss of generality, assume
  that $G$ is connected.  If $|V(G)|< 4$, then the result is trivial, so
  we can assume $|V(G)|\ge 4$.  We now proceed by induction on $|V(G)|$.

  Let $T=T(G)$ be a 2-tree with vertex set $V(G)$ and that contains $G$.
  Apply \lemref{dujwood} to find the vertex set $S$ and the vertex $u$.
  Let $x$ and $y$ be the neighbours of $u$ in $T\setminus S$. Now apply
  induction to find a straight-line $d_4$ obstacle embedding of the
  graph $G'$ whose vertex set is $V(G')=V(G)\setminus S$ and whose edge
  set is $E(G')=E(G\setminus S)\cup\{ux,uy\}$.

  Now observe that, since $u$ has degree 2 in $G'$ and the edges $ux$
  and $uy$ are in $G'$, this embedding does not contain any monotone path
  of the form $uxw$ or $uyw$ for any $w\in V(G)\setminus\{u,x,y\}$.
  Therefore, if we place the vertices in $S$ sufficiently close to $u$,
  we will not create any monotone path of the form $ayw$ or $axw$ for
  any $a\in S$ and any $w\in V(G)\setminus \{u,x,y\}$.  What remains
  is to show how to place the elements of $S$ in order to avoid unwanted
  monotone paths of the form $uay$, $uax$, or $aub$ for any $a,b\in S$.
  There are three cases to consider (see \figref{2-tree-proof}):

  \begin{figure}
    \begin{tabular}{cccccc}
       \multicolumn{6}{c}{\includegraphics{figs/2tree-1}} \\
       \multicolumn{6}{c}{1(a)} \\
       \multicolumn{3}{c}{\includegraphics{figs/2tree-2}} & 
       \multicolumn{3}{c}{\includegraphics{figs/2tree-3}} \\
       \multicolumn{3}{c}{2(a)} &
       \multicolumn{3}{c}{2(c)} \\
       \multicolumn{2}{c}{\includegraphics{figs/2tree-4}} &
       \multicolumn{2}{c}{\includegraphics{figs/2tree-5}} &
       \multicolumn{2}{c}{\includegraphics{figs/2tree-6}} \\
       \multicolumn{2}{c}{3(a)} &
       \multicolumn{2}{c}{3(b)} &
       \multicolumn{2}{c}{3(c)} 
    \end{tabular}
  \end{figure}
  
  \begin{enumerate}

  \item  $x\in Q_i(u)$ and $y\in Q_{i+2}(u)$ for some
  $i\in\{0,\ldots,3\}$. Without loss of generality, assume that
  $Q_{i+3}(u)$ does not intersect the segment $xy$. Then we can embed the
  elements of $S$ in $Q_{i+3}$ without creating any new monotone paths.

  \item $x,y\in Q_i(u)$ for some $i\in\{0,\ldots,3\}$. There are two subcases:
    \begin{enumerate}
    \item At least one of $ux$ or $uy$ is in $E(G)$. Suppose $ux\in E(G)$.  Then we embed $S_x$ in $Q_i(u)$ and embed $S_y$ in $Q_{i+1}(u)$.  The only monotone paths this creates are of the form $uax$ with $a\in S_x$, which is acceptable since $ux\in E(G)$.
    \item Neither $ux$ nor $uy$ is in $E(G)$. In this case, we embed all of $S$ in $Q_{i+2}(u)$.  This does not create any new monotone paths.
  \end{enumerate}

  \item $x\in Q_i(u)$ and $y\in Q_{i+1}(u)$ for some $i\in\{0,\ldots,3\}$.  We have two subcases to consider:
  \begin{enumerate}
    \item $|\{ux,uy\}\cap E(G)|=1$.  In this case, assume $ux\in E(G)$. Then we embed the vertices of $S_x$ in $Q_i(u)$ and we embed the vertices of $S_y$ in $Q_{i+3}(u)$.  The only monotone paths this creates are of the form $uax$ with $a\in S_x$, which is acceptable since $ux\in E(G)$.
    \item $|\{ux,uy\}\cap E(G)|=2$.  In this case we embed the vertices of $S_x$ in $Q_i(u)$ and we embed the vertices of $S_y$ in $Q_{i+1}(u)$.  The only monotone paths this creates are of the form $uax$ with $a\in S_x$ and $uby$ with $b\in S_y$, which is acceptable since $ux,uy\in E(G)$.
    \item $|\{ux,uy\}\cap E(G)|=0$.  In this case, we embed all of $S$ into $Q_{i+2}\cup Q_{i+3}$.  This does not create any new monotone paths.

    {:.center}
    ![2-tree case 2.1](images/2tree-4.svg)
    ![2-tree case 2.1](images/2tree-5.svg)
    ![2-tree case 2.1](images/2tree-6.svg)
\end{enumerate}
\end{enumerate}
This completes the proof.
\end{proof}

$(\R^2,d_4,\rho_4)$ obstacle representation.







\section{General Graphs}







\bibliographystyle{plain}
\bibliography{geoobs}

\end{document}


